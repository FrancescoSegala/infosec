\documentclass[a4paper,11pt]{article}
\usepackage{graphicx}
\usepackage{amsmath}
\usepackage{algorithm}
\usepackage{float}
\usepackage{hyperref}
\usepackage{listings}
\usepackage[noend]{algpseudocode}
\graphicspath{{./}}

\usepackage{graphicx}

\begin{document}

\title{Exercise 2 }
\author{Francesco Segala 3521885 Manos ??????? Sergio ???????? }
\maketitle

\section{The Assignement}
Submit the encryption key you used, the decrypted text, and the source of your program.
\subsection{Mapping key}

A substituction algorith is applied to a web page, our goal is to find the key of this algorithm and to
obtain the orignial text.\\
To cope with this we follow the insight of the assignment in such a way that we can easily find the mapping function
that maps the plaintext to the encrypted one.\\
We come up with this mapping key:

\framebox[\linewidth]{ uoieazyxwvtsrqpnmlkjhgfdcb }\par

 With the same logic for capital letters, ignoring all other symbols.

\subsection{The code}
Once we found the mapping key the algorithm to build is trivial :


\begin{lstlisting}[language=Python]

mapped_key="uoieazyxwvtsrqpnmlkjhgfdcb"
mapped_key_capital=mapped_key.upper()
mapped_key=mapped_key+mapped_key_capital
origin_key="abcdefghijklmnopqrstuvwxyz"
origin_key_capital=origin_key.upper()
origin_key=origin_key+origin_key_capital

diz={}
for index in range(0,len(mapped_key )):
    diz[mapped_key[index]] = origin_key[index]

encrypted_txt="path for the encrypted file"
result="path of the result file"

def decrypt(in_path, out_path):
    encr=open(in_path,"r").read()
    decrypted=open(out_path,"w+")
    for letter in encr :
        if letter.isalpha() :
            resletter=diz[letter]
            decrypted.write(resletter)
        else:
            decrypted.write(letter)

decrypt(encrypted_txt,result)
\end{lstlisting}

\subsection{Plaintext}

so we finally post the decripted page at the link \href{https://security.rug.nl/infosec/intro/2017.enc}{intro/2017.enc} as follow:
\begin{lstlisting}
9 common security awareness mistakes (and how to fix them)

To err is human, but to err in cyber security can cause major damage to an
organization. It will never be possible to be perfect, but major improvement
is possible, just by being aware of some of the most common mistakes and their
consequences.



1. Falling for phishing: One of the most common mistakes.
2. Unauthorized application or cloud use, known as shadow IT.
3. Weak or misused passwords
4. Remote insecurity: This is the common practice of transferring files
   between work and personal computers
5. Disabling security controls: This is usually done by users with
   administrative privileges, to make things easier for employees
6. Clueless social networking
7. Poor mobile security
8. Too many privileges
9. Failure to update or patch software

See also http://www.csoonline.com/article/2877259/security-awareness/
            nine-common-security-awareness-mistakes-and-how-to-fix-them.html

\end{lstlisting}




\end{document}
